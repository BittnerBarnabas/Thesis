\section{User documentation}
\subsection{Problems solved by new modules}
\par The four new modules created for Clang-tidy aim to address four fairy distinctive and inefficient programming patterns. 
\subsubsection{Inefficient String Concatenation\cite{clang_tidy_string_concat}}
\par The first problem involves the standard library's string class. This class, more precisely \\\verb|std::basic_string<char>|, is the default implementation of a modifiable high-level character sequence in the language. It provides several functions which can modify, transform the underlying characters. \medskip
\par The problem arises when one wants to concatenate two or more strings. There are several methods achieving the concatenation: \verb|operator+|, \verb|operator+=| and \verb|.append| member function. The problem arises when one wants to include the original string in the concatenation with a structure similar to \verb|a = a + b;| which is highly inefficient. In Listing:\ref{lst:ineff_string_concat} the concatenation using \verb|operator+| is causing a noticeable performance overhead in the application.  
\begin{lstlisting}[language=c++, frame=single ,caption={Highly inefficient code}, label={lst:ineff_string_concat}]
std::string a("Foo"), b("Baz");
for (int i = 0; i < 20000; ++i) 
{
	a = a + "Bar" + b;
}
\end{lstlisting}
\par This program could be refactored into a much faster code using either \verb|operator+=| or \verb|.append()| member function as shown in Listing:\ref{lst:ineff_string_solved}.
 \begin{lstlisting}[language=c++, frame=single ,caption={A more efficient version}, label={lst:ineff_string_solved}]
std::string a("Foo"), b("Baz");
for (int i = 0; i < 20000; ++i) 
{
	a.append("Bar").append(b);
}
 \end{lstlisting}
\subsection{Installing the program}
\subsubsection{Windows}
\subsubsection{Linux}
\subsubsection{OS X}
\subsection{Using the program}