\section{User documentation}
\subsection{Problems solved by new modules}
\par The four new modules created for Clang-tidy aim to address four fairy distinctive and inefficient programming patterns. 
\subsubsection{Inefficient String Concatenation\cite{clang_tidy_string_concat}}
\par The first problem involves the standard library's string class. This class, more precisely \\\verb|std::basic_string<char>|, is the default implementation of a modifiable high-level character sequence in the language. It provides several functions which can modify, transform the underlying characters. \medskip
\par The problem arises when one wants to concatenate two or more strings. There are several methods achieving the concatenation: \verb|operator+|, \verb|operator+=| and \verb|.append()| member function. The problem arises when one wants to include the original string in the concatenation with a structure similar to \verb|a = a + b;| which is highly inefficient. In Listing:\ref{lst:ineff_string_concat} the concatenation using \verb|operator+| is causing a noticeable performance overhead in the application.  \\
\begin{lstlisting}[language=c++, frame=single ,caption={Highly inefficient code}, label={lst:ineff_string_concat}]
std::string a("Foo"), b("Baz");
for (int i = 0; i < 20000; ++i) 
{
	a = a + "Bar" + b;
}
\end{lstlisting}
\par This program could be refactored into a much faster code using either \verb|operator+=| or \verb|.append()| member function as shown in Listing:\ref{lst:ineff_string_solved}.
 \begin{lstlisting}[language=c++, frame=single ,caption={A more efficient version}, label={lst:ineff_string_solved}]
std::string a("Foo"), b("Baz");
for (int i = 0; i < 20000; ++i) 
{
	a.append("Bar").append(b);
}
 \end{lstlisting}
 \par This check doesn't involve a FixItHint because the solution cannot be generalised 
 \subsubsection{Inefficient Stream Use}
 \par The second problem involves the standard library's \verb|operator<<| and \verb|std::ostream|. In C++ you can use 
\verb|std::cout| to print to the console output, which is really helpful for logging or just this is the way the program communicates with the user, for instance Clang-tidy or Clang itself.\medskip
\par There are two issues with using the standard way of interaction with the console. First is when using single characters yet annotating them with \verb|""| (double quotes), for example: \verb|"a"|. It's quite inefficient when you put such constructed characters to the stream. Instead one should construct single characters as \verb|'a'|.\medskip \begin{lstlisting}[language=c++, frame=single ,caption={Sightly inefficient way of streaming characters}, label={lst:ineff_stream_pro}]
std::cout << "a" << "b" << "c";
\end{lstlisting}
\begin{lstlisting}[language=c++, frame=single ,caption={A generally more efficient version}, label={lst:ineff_stream_solved}]
std::cout << 'a' << 'b' << 'c';
\end{lstlisting}
\par The other problem which raises more concern about the performance is streaming multiple \\\verb|std::endl| after each other. Unnecessary use of \verb|std::endl| can potentially cause massive performance hogs in an application especially if it's a library which is intended to be used by others. For example the code in Listing:\ref{lst:ineff_stream_prob2} can be rewritten in a highly more efficient form as in Figure:\ref{lst:ineff_stream_solved2}
\begin{lstlisting}[language=c++, frame=single ,caption={A really slow way to print newlines}, label={lst:ineff_stream_prob2}]
std::cout << std::endl << std::endl << std::endl;
\end{lstlisting}
\begin{lstlisting}[language=c++, frame=single ,caption={A more efficient version}, label={lst:ineff_stream_solved2}]
std::cout << '\n' << '\n' << std::endl;
\end{lstlisting}
\par This check contains FixItHints for correcting the aforementioned issue. It replaces \verb|""| with \verb|''| and also rewrites multiple \verb|std::endl| function calls in a stream, except for the last one. 
 \subsubsection{Shared Pointer Conversion}
 \par This problem is relevant to \verb|std::shared_ptr| in the standard library. A \verb|shared_ptr| is a smart pointer, therefore it cannot dangle (point to invalid memory address), which allows other \\\verb|shared_ptr|s to point at the same object as the first pointer, as opposed to \verb|std::unique_ptr| which doesn't allow multiple owners of the object it points to. This makes it slightly more overweight than the \verb|unique_ptr| because each pointer has to know whether it's the last one keeping a reference to the object, and if so when it's destructed it needs to free the resources allocated for the object it points to. \medskip
 \par In C++ where classes are involved with the use of polymorphism a derived class' instance can always be converted to a base class' one. With this casting operation and \verb|shared_ptr|'s way of knowing how many references there are for the object, passing \verb|shared_ptr|s via function arguments where there is implicit (or explicit) casting used is really inefficient in terms of performance. \medskip
 \begin{lstlisting}[language=c++, frame=single ,caption={Inefficient implicit cast}, label={lst:ineff_shared_prob}]
 class A {};
 class B : A {};
 void f(std::shared_ptr<A>){}
 int main()
 {
 	auto ptr = std::make_shared<B>();
 	f(ptr);
 } 
\end{lstlisting}
\par In Listing:\ref{lst:ineff_shared_prob} there is a small example program demonstrating the code which can be greatly improved just by getting the reference of the underlying object of the pointer, as seen in Listing:\ref{lst:ineff_shared_solved}. This program is about 2.5 times faster than the one before. 
\begin{lstlisting}[language=c++, frame=single ,caption={A much faster version}, label={lst:ineff_shared_solved}]
class A {};
class B : A {};
void f(const A&){}
int main()
{
	auto ptr = std::make_shared<B>();
	f(*ptr.get());
}
\end{lstlisting}
\par For this check there's no FixItHint option, because there are different ways of refactoring the inefficient structure and if the user accessed the \verb|shared_ptr|'s API then it would be nearly impossible to find a simple automatic solution for refactoring such code. 
\subsubsection{Default Container Initialization}
\par The last problematic code pattern addressed involves the standard library's default containers, namely \verb|std::vector|, \verb|std::set|, \verb|std::deque| and \verb|std::map|. In C++ when initialising containers one can initialise them and add the elements afterwards, which is slightly less efficient than initialising the containers in the expression they are declared. \medskip
\begin{lstlisting}[language=c++, frame=single ,caption={Inefficient way of creating containers}, label={lst:ineff_cont_prob}]
std::vector<int> vec;
vec.push_back(3);
vec.emplace_back(1.2);
vec.push_back(1.0);
\end{lstlisting}
\par As in Listing:\ref{lst:ineff_cont_prob} the Initialization of an \verb|std::vector| could be done in a more efficient way, with less lines of code. In Listing:\ref{lst:ineff_cont_solv}  the refactored version needs some explicit conversions to work. 
\begin{lstlisting}[language=c++, frame=single ,caption={A faster way of initialising}, label={lst:ineff_cont_solv}]
std::vector<int> vec{3, (int) 1.2, (int) 1.0};
\end{lstlisting}
\subsection{Performance benchmarks}
\par In this section, I am showing some performance benchmarks of code which are affected by the problems each check fixes and compare these results to the results of the improved, refactored code. Each test is executed in a well isolated environment to highlight the execution times of only the code we are interested in. 
\subsubsection{Shared Pointer Conversion}
\par After applying the suggested fixes by the Shared Pointer Conversion checker we see a significant improvement in speed compared to the previous version. \medskip
\par If we try to fit a line over the points with linear regression we get $y=0.0002632x+0.0000059$ with $R^2=0.98$ for the refactored one and $y = 0.002938x+0.0075$ with $R^2=1$ for the original one. From these line equations we can see that the original is about 10 times steeper than the refactored one. 
\begin{figure}[H]
	\caption{Shared pointer conversion benchmark}
	\includegraphics[scale=0.7]{images/shared_ptr_performance.pdf}
\end{figure}
\subsubsection{Inefficient String Concatenation}
\par When benchmarking the speed of concatenating strings in a loop the results were surprising. As it turns out naively concatenating string in a loop is has a quadratic time complexity, yet if we modify on the operation as the check suggests we get a linear time complexity that is much faster than the original version. We can fit a quadratic polynomial $y=0.0002033x^2+0.0086x-0.00085$ with $R^2=1.0$ to the original and a linear $y=0.0001347x+0.00380$ with $R^2=0.84$ to the refactored. 
\begin{figure}[H]
	\caption{String concatenation benchmark}
	\includegraphics[scale=0.7]{images/string_concat.pdf}
\end{figure}
\subsubsection{Container Default Initialization}
\par Benchmarking code that used an inefficient way of initialising an \verb|std::vector| on which the check was run with FixItHints enabled, the automatic refactoring provided a much faster code than the original one. Fitting lines on both samples gives us $y=0.007635x+0.0038$ with $R^2=1.0$ line for the original and $y=0.002486x+0.0046$ with $R^2=1.0$ for the refactored one, resulting in a 3 times faster code.
\begin{figure}[H]
	\caption{Container Initialization benchmark}
	\includegraphics[scale=0.7]{images/container_init_performance.pdf}
\end{figure}
\subsubsection{Inefficient Stream Use}
\par Benchmarking this check was divided into two categories because it warns about two distinct, yet related, performance problems. \medskip 
\par First I measured the performance gains of refactoring \verb|<< "a"| type of calls to \verb|<< 'a'| types. The surprising result that I got was that there was no significant increase in performance when applied the FixItHints from the check only a marginal benefit. Performing linear regression analysis on the samples we get $y=0.004748x+0.0061$ with $R^2=1.0$ for the original and a slightly smoother $y=0.004596x+0.004596$ with $R^2=1.0$ for the refactored code.
\begin{figure}[H]
	\caption{Char conversion benchmark}
	\includegraphics[scale=0.7]{images/stream_char_performance.pdf}
\end{figure}
\par The other type of tests were conducted on using multiple \verb|std::endl| on a single stream. The correction of this problem was also automatic, and the results show a 2 times increase in performance after applying the FixItHints. Fitting lines to the original gives a $y=0.01758x+0.059$ with $R^2=1.0$ and a $y=0.008256x+0.001$ with $R^2=1.0$ on the refactored. 
\begin{figure}[H]
	\caption{Multiple end-line benchmark}
	\includegraphics[scale=0.7]{images/stream_endline_performance.pdf}
\end{figure}
\subsection{Installing the program}
Installing the program with the new modules is fairly an easy process, but it varies from operating system to operating system.
\subsubsection{Windows}
\par Installing the tool on Microsoft Windows is a little bit more complicated than on Linux. There are several paths that you can take to set up a Linux environment inside Windows or you can use a native Windows approach. 
\paragraph{MinGW} is a project for Windows which aims to provide a minimalist GNU development environment\cite{mingw_homepage}. This project doesn't aim to provide a POSIX run-time environment\cite{mingw_homepage}, but wants to simplify how one can use the C run-time Microsoft provides and other language run-times \medskip
\par MinGW currently features the following applications:
\begin{enumerate}
	\item A GNU Compiler (\verb|gcc|) port for Windows
	\item GNU Binutils which contains a set of utility programs
	\item MSYS(Optional)
\end{enumerate}
\par MSYS is a collection of different utility programs from GNU project. It's an optional complementary designed for MingGW\cite{mysis_page}. Its goal is to simplify interacting with UNIX tools, and also to provide a better command-line terminal application for Windows by providing Bash. \medskip
\par To install the program with MinGW, first you have to download the latest MinGW run-time from the original site: \url{http://www.mingw.org/}. Following this guide one should be able to install MinGW, MSYS, subversion and cmake on their Windows PC: \url{http://ingar.satgnu.net/devenv/mingw32/base.html}. \medskip
\par With these tools you can continue with the second step of Linux installation guide, following the steps carefully. \medskip
\paragraph{Cygwin}
\par An other option to installation is using Cygwin which is an environment aiming at creating a POSIX compatible layer in Windows. \medskip
\paragraph{WSL}
\par also known as Windows Subsystem for Linux is an optional feature from Microsoft on Windows 10 Desktop operating systems. The subsystem enables the execution of native \verb|ELF64| binaries\cite{wsl_overview} on Windows, without a need of a virtual machine or third party layer. \medskip
\par WSL works by using a so-called Pico process which was developed at Microsoft under project Drawbridge\cite{project_drawbridge} to enable a way of application sandboxing "with minimal kernel API surface"\cite{project_drawbridge}. This Pico process wraps the native Linux application while mapping all the Linux kernel calls to NT kernel calls hence providing the seamless functionality. 
\par 
\subsubsection{OS X}
\par On macOs (formerly OS X) the installation is fairly straightforward, we just need some preparation before we can continue with the Linux guide in \ref{linux}. First of all because macOs is a Unix type system (its roots are in BSD) working in a Terminal on macOs is quite familiar experience with a native Linux one, both having Bash as the default terminal application. But before we can deep dive into the Linux steps we need to install a package manager, because macOs doesn't have one out of the box.
\par We will install the Homebrew\cite{homebrew_homepage} package manager, as this supports the most features and it's the most supported one. Open a terminal and type: 
\begin{lstlisting}[language=bash, frame=single]
user@host:$ ruby -e "$(curl -fsSL https://raw.githubusercontent.com/Homebrew/install/master/install)"
\end{lstlisting}
\par This will install the whole homebrew system, and after that you will be able to install packages with the \verb|brew install| command. Now you should install Xcode from the AppStore as it will install a lot of useful packages to the system, for instance \verb|gcc|, \verb|ld|, \verb|clang|.
\par Now we should install the requirements to download LLVM and Clang:
\begin{lstlisting}[language=bash, frame=single]
user@host:$ brew install svn, cmake
\end{lstlisting}
\par After successfully installing these applications you can head to the second point of the Linux guide to finalise the installation process.
\subsubsection{Linux}\label{linux}
\par Before we start installing the tool on a Linux system, we have to install some basic applications needed to download and compile the program. In the Linux world there are a lot of package manager applications, for instance there is (on default) \verb|apt-get| on Debian-based systems, \verb|Pacman| on Arch Linux-based systems, \verb|rpm| on RedHat-based systems and \verb|Portage| on Gentoo-based systems. To install a package on these different systems use the following commands (supposing one have superuser privileges):
\begin{lstlisting}[language=bash, frame=single]
user@host:$ apt-get install <package-name>
user@host:$ pacman -S <package-name>
user@host:$ yum install <package-name>
user@host:$ emerge <package-name>
\end{lstlisting}
\par From now on, I will show the commands only on Debian-based systems, but translating the given instructions to a different Linux flavour shouldn't be hard. \medskip 
\par Firstly we have to install subversion(\verb|svn|) on the machine which is a version control system. Also we will need \verb|cmake| which can generate Linux makefiles to make it easy to compile the application. To install them, type:
\begin{lstlisting}[language=bash, frame=single]
user@host:$ sudo apt-get install svn, cmake
\end{lstlisting}
\par We need subversion because the LLVM main project and Clang is in an svn repository over the internet. Then we need to clone (download) LLVM first, then clone Clang into the previously cloned LLVM project. Navigate to a comfortable directory where you want your project to reside and clone the repositories:
\begin{lstlisting}[language=bash, frame=single]
user@host:$ cd /directory/where/your/project/should/be
user@host:$ svn co http://llvm.org/svn/llvm-project/llvm/trunk llvm
user@host:$ cd llvm/tools
user@host:$ svn co http://llvm.org/svn/llvm-project/cfe/trunk clang
user@host:$ cd ../..
\end{lstlisting}
\par If you are comfortable with the default C++ standard library you can skip the next step, otherwise if you want to use newer libc++ standard library implementation, proceed with the following:
\begin{lstlisting}[language=bash, frame=single]
user@host:$ cd llvm/projects
user@host:$ svn co http://llvm.org/svn/llvm-project/libcxx/trunk libcxx
user@host:$ cd ../..
\end{lstlisting}
\par Now you have the framework required for the \verb|extra| module. Copy the \verb|extra| directory from the location where the thesis is, to the Clang project:
\begin{lstlisting}[language=bash, frame=single]
user@host:$ cp /dir/to/thesis/location/extra llvm/clang/tools/
\end{lstlisting}
\par Now we are ready to compile the whole project. Create a build directory outside of the project structure (in-tree builds are not supported\cite{clang_getting_started}), then generate Unix Makefiles and compile the project. We pass \verb|-DCMAKE_BUILD_TYPE=RELEASE| argument to \verb|cmake| to set it to release build type.
\begin{lstlisting}[language=bash, frame=single]
user@host:$ mkdir build
user@host:$ cd build
user@host:$ cmake -G "Unix Makefiles" ../llvm -DCMAKE_BUILD_TYPE=RELEASE
user@host:$ make
\end{lstlisting}
\par Cmake supports many modern IDEs, see its documentation for different generators\cite{cmake_generator_doc}. \medskip
\par After the whole project compiled successfully you will find the built executable files in \verb|build/bin| directory.
\subsection{Using the program}
\par After the program has been successfully installed on the given system we can begin using it to run the modules on distinct C++ files or on whole projects. 
\subsubsection{CLI}
\par When it comes to running the application from the command line, its important to distinguish running the tool on different, standalone source-files and on a whole C++ project. \medskip
\par Running on standalone source files is the easiest method when only couple of files are involved which don't require any other library to compile. To run with the default checks enabled on a file named \verb|source.cpp|:
\begin{lstlisting}[language=bash, frame=single]
user@host:$ clang-tidy ./source.cpp
\end{lstlisting}
\par If you want to specify which checks should be used when analysing you can use the \verb|-checks=| flag, and after that you can use the checker name to add, and a \verb|-| checker name to exclude. You can use \verb|*| to refer to all the checks:
 \begin{lstlisting}[language=bash, frame=single]
 user@host:$ clang-tidy -checks=-*,performance-inefficient-stream-use ./source.cpp
 \end{lstlisting}
 \par This runs only the Inefficient String Use check. If you need to pass arguments to the compiler behind Clang-tidy you can use the \verb|--| flag and then just pass the arguments. For instance this code will run Clang-tidy with the default checks enabled, passing \verb|-std=c++14| argument to the compiler:
 \begin{lstlisting}[language=bash, frame=single]
 user@host:$ clang-tidy ./source.cpp -- -std=c++14
 \end{lstlisting}
 \par To specifically run the four modules involved in this thesis you can pass these arguments to \verb|-checks| parameter:
 \begin{itemize}
 	\item \verb|performance-inefficient-shared-pointer-reference|
 	\item \verb|performance-inefficient-stream-use|
 	\item \verb|performance-inefficient-string-concatenation|
 	\item \verb|performance-container-default-initializer|
 \end{itemize}
\par There are several options running the tool on a whole project. The fundamental principle to run the tool on multiple files, is to have a JSON Compilation database, which is "a format for specifying how to replay single compilations independently of the build system"\cite{compilation_db}. 
\par Generating a JSON Compilation database can be done several ways: 
\begin{itemize}
	\item using \verb|cmake| and passing \verb|-DCMAKE_EXPORT_COMPILE_COMMANDS=ON|
	\item with Ninja build system's \verb|compdb| tool\cite{ninja_build_system}
	\item since version 4.0.0 Clang supports generating it with \verb|-MJ| argument\cite{clang_release_notes}
\end{itemize}
\subsubsection{GUI}