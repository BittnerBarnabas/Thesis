\section{Developer documentation}
\subsection{Detailed description of the solved problems}
\par In this section, I am going to present the technical background of the mentioned problems, why they are inefficient and what can we do to refactor them. I am referencing the \verb|N4567|\cite{cpp_standard} version Working Draft of the C++ standard, which contains every C++14 features and also freely available.
\subsubsection{Shared Pointer Conversion Check}
\par According to the C++ standard the \verb|memory| header has to contain a templated class \verb|shared_ptr| with the signature \verb|template<class T> class shared_ptr|\cite[\S20.8.2.2]{cpp_standard} \medskip
\par This class is responsible for storing a standard C++ pointer and implementing shared ownership of this raw pointer. This means that the last \verb|shared_ptr| object is responsible for the destruction of the underlying pointer. \medskip
\par The way \verb|shared_ptr| achieves the shared ownership is by a class-wide static variable (\verb|use_count()|) which is incremented in each construction and decremented in each destruction and if \verb|use_count()|  reaches zero in a destruction then the underlying pointer shall be destroyed either by default \verb|delete| keyword or by a user-supplied custom deleter. The modification of the reference count has to be an atomic process and should not be modified simultaneously\cite[\S20.8.2.2 4.]{cpp_standard}. \medskip
\par These properties could make the \verb|shared_ptr| a hidden source of performance hogs when implicit temporary object creation is involved.  Such as the case when inherit is used. \medskip
\par The standard says that a pointer to a derived class shall always be implicit convertible to a pointer to the base class\cite[\S4.10 3.]{cpp_standard}. Also in C++ objects generated  by template instantiations are not covariant, they are invariant in fact, meaning there is no implicit conversion in this example, but a compile time error:
\begin{lstlisting}[language=c++, frame=single]
struct A {}; struct B : A {};
std::vector<B*> vec1;
std::vector<A*> vec2;
vec2 = vec1;
\end{lstlisting}
\par Yet to make \verb|shared_ptr<A>| covariant, to preserve to resemblance to a raw pointer, the standard defines constructors with signature
\\ \verb|template<class Y> shared_ptr(const shared_ptr<Y>& r) noexcept;|\\
\verb|template<class Y> shared_ptr(shared_ptr<Y>&& r) noexcept;| \\
\cite[\S20.8.2.2.1]{cpp_standard}
which should be eligible for overload resolution if and only if \verb|T*| is convertible to \verb|Y*|. \medskip
\par This entails that a \verb|shared_ptr<A>| shall be implicit convertible to \verb|shared_ptr<B>| if \verb|A| is a derived class from \verb|B| and \verb|B| is accessible and unambiguous. But this conversion is costly, because of the reference counting, which in this case decrements the reference count in \verb|shared_ptr<Derived>| and increments in \verb|shared_ptr<Base>|. This increment, decrementing is unnecessary and resource consuming. \medskip
\par To solve this problem, the easiest method is to refactor the function which expects a \verb|shared_ptr<Base>|, to expect a \verb|const Base&| or \verb|const B*| and call the function with either \verb|*ptr.get()| or \verb|ptr.get()|.
\subsubsection{Inefficient Stream Use Check}
\par This problem consists of two parts. Firstly what makes it marginally inefficient to stream characters as strings. The standard defines that the hierarchy of the classes which handle Input/Output should look like in Figure:\ref{cpp_io_hierarch}.
\begin{figure}[H]
	\caption{C++ IO hierarchy}
	\centering
	\includegraphics[scale=0.6]{images/cpp_io_structure.pdf}
	\label{cpp_io_hierarch}
\end{figure}
\par In \verb|basic_ostream| there are several overloads of \verb|operator<<| however we are only interested in two particular overload:
\begin{itemize}
	\item \begin{verbatim}
		template<class traits>
basic_ostream<char,traits>& operator<<(basic_ostream<char,traits>&,
char);
	\end{verbatim}
	\item \begin{verbatim}
template<class traits>
basic_ostream<char,traits>& operator<<(basic_ostream<char,traits>&,
const char*);
\end{verbatim}
\end{itemize}\cite[\S27.7.3.6.4]{cpp_standard}
These overloads enable the \verb|basic_ostream| object to accept both \verb|<< "A"| and \verb|<< 'A'| structures. \medskip
\par These two overloads both achieve the same goal, however performance-wise they are not the same, since \verb|"A"| structure will be represented by \verb|const char[2]|\footnote{strings are closed by null bytes in C++ so it would look like 'A'\textbackslash 0} in the code while \verb|'A'| will be only a single instance \verb|char|. And since arrays can be implicitly converted to pointers\cite[\S4.2]{cpp_standard} the former version will call the overload with \verb|const char*| in its signature.
\par The only problem with \verb|const char*| is that the length of the string is not known, so it must be processed until the tailing null-byte is found. On the contrary, when using \verb|char| the length is known, so no extra computation is required. However since modern CPUs are quite fast this extra computational complexity barely affects the observable performance, but in an embedded or in other time-critical even this small performance gain is significant. \medskip
\par The other inefficiency related to \verb|basic_ostream| are the multiple calls to \verb|endl| function which is defined as\cite[\S27.7.3.8]{cpp_standard}:
\begin{verbatim}
template <class charT, class traits>
  basic_ostream<charT,traits>& endl(basic_ostream<charT,traits>& os);
\end{verbatim}
To be able to apply \verb|operator<<| on functions the standard defines the following\cite[\S27.7.3.6]{cpp_standard}:
\begin{verbatim}
basic_ostream<charT,traits>& operator<<(
   basic_ostream<charT,traits>& (*pf)(basic_ostream<charT,traits>&));
\end{verbatim}
which enables \verb|operator<<| to take a function, that returns \verb|basic_ostream&|, as an argument. \medbreak
\par With this information one could think that streaming more than one \verb|std::endl| is equivalent to streaming \verb|'\n'| characters. However the standard also states that the \verb|endl| function has to call \verb|flush|\cite[\S27.7.3.8]{cpp_standard} which flushes the output buffer, which is a time consuming procedure. The actual performance overhead caused by flushing the output buffer is operating system dependent, but in general it can be categorised as a performance hog, if used repeatedly.
\subsubsection{Inefficient String Concatenation Check}
\par As the C++ standard defines the Strings library it provides several ways of concatenating two strings together, \verb|a = a + b;|, \verb|a += b;| and \verb|a.append(b);|. \medskip
\par The first one requires two operators to be declared, the first is \verb|operator+| for \verb|basic_string| \cite[\S21.4.8.1]{cpp_standard}:
\begin{verbatim}
template<class charT, class traits, class Allocator>
basic_string<charT,traits,Allocator>
  operator+(const basic_string<charT,traits,Allocator>& lhs,
            const basic_string<charT,traits,Allocator>& rhs);
\end{verbatim}
\par And also the \verb|basic_string|'s relevant constructor:
\begin{verbatim}
basic_string(basic_string&& str) noexcept;
\end{verbatim}
\par The other two ways of concatenating is with a designated member function: \cite[\S21.4.6.1]{cpp_standard} and \cite[\S21.4.6.2]{cpp_standard}
\begin{verbatim}
basic_string&  operator+=(const basic_string& str);
basic_string&  append(const basic_string& str);
\end{verbatim}
\par Where \verb|operator+=| just calls \verb|append|. \verb|append| then does the process of concatenating the strings. The standard does not say how the implementation should behave in the \verb|append| operation, but an industrial grade standard implementation should use the possibility of simply copying (moving if applicable) the characters of the \verb|append| argument to the receiver's buffer if it fits. Otherwise the buffer should be increased and the whole temporary object copied. \medskip
\par This way it would be more efficient to use \verb|append| member function instead of creating a new string with containing the exact copy of itself and then assigning it to itself.
\subsubsection{Default Container Initialization Check}
\par The C++ standard defines standard containers which are able to hold collections of objects. The containers which are defined and we are interested in are: \verb|std::vector|, \verb|std::set|, \verb|std::deque| and \verb|std::map|. To put elements in these containers one can define them with the default constructor and later add elements one by one with insertion operations. \medskip
\par Each insertion is a call of a member method for instance \verb|push_back()|, \verb|insert()| and \verb|emplace_back()|. To use these methods right after creating the container is inefficient because the standard mentions that every mentioned container must have a constructor which takes an initializer list as an argument:
\begin{itemize}
	\item \verb|vector(initializer_list<T>, const Allocator& = Allocator());| where \verb|T| is the Type of the objects in the container \cite[\S23.3.6.2]{cpp_standard}
	\item \verb|map(initializer_list<value_type>...);| where \verb|value_type| is a pair of the key and value type of the \verb|map| \cite[\S23.4.4.2]{cpp_standard}
	\item \verb|set(initializer_list<value_type>...);| where \verb|value_type| is the type of the \verb|set|'s key \cite[\S23.4.6.2]{cpp_standard}
	\item \verb|deque(initializer_list<T>, const Allocator& = Allocator());| where \verb|T| is the Type of the objects in the container\cite[\S23.3.3.2]{cpp_standard}
\end{itemize}
\par The more efficient form would be list-initialising the containers in place with the declaration by passing an \verb|initializer_list| to the constructor. But the list-initialisation raises concerns when implicit casting is involved. The standard states "If a narrowing conversion \dots is required to convert any of the arguments, the program is ill-formed"\cite[\S8.5.4 3.6]{cpp_standard}. \medskip
\par A narrowing conversion is an implicit conversion of numerical types where there is a possibility of over- or underflow when converting to the destination type. For a more detailed description see \cite[\S8.5.4 7]{cpp_standard}. To remove implicit narrowing conversions we need to introduce explicit casting. This checker uses C-style casts, which in the future should be replaced with a safer \verb|static_cast|. \medskip
\par The standard also ensures that the elements in a list-initializer are constructed from left to right order\cite[\S8.5.4.4]{cpp_standard}, this way not violating the order they were originally added to the container.
\subsection{Implementations}
\par In this section I am going to cover the implementation details of each module, covering the possible pitfalls and issues with implementing a standalone Clang-tidy module.
\subsubsection{General implementation details}
\par Each check have at least two methods, which are inherited from the \verb|ClangTidyCheck| base class. The first method is \verb|registerMatchers| which acts as a set-up phase. The actual AST matcher structure should be assembled in this function, adding all components. Also, the callback method is specified in this segment, more precisely the object on which the callback method is called, this is usually \verb|this|. In this method, the developer has access to a \verb|MatchFinder| object which controls the matchers. To add an AST matcher to the \verb|MatchFinder|, the member function \verb|addMatcher()| can be used, that takes an AST matcher and a callback object as parameters. \medskip
\par The other method is the callback function named \verb|check|. This method gets called for every match on the AST by the checkers registered beforehand in the \verb|registerMatchers| method. In this function, the developer has access to virtually everything from the actual source code to the AST nodes, which can be queried from a \verb|MatchResult| object.
\subsubsection{Inefficient String Concatenation Check}
\par Since Clang 4.0.0\cite{string_concat_release} this module is part of the official Clang svn repository and had been referenced in a guide to improve developer workflow productivity\cite{string_concat_ref}.
\par When implementing this checker the greatest challenge was matching such \\\verb|operator=| operator calls which have the left-hand side variable on the right-hand side also. Initially, there were two options, either filter on the AST matcher's side or match everything and then filter in the callback function. \medskip
\par Filtering in the AST matcher is the preferred method because it's less resource intensive and clearer. This is handled by first searching for an \verb|operator=| with a \verb|basic_string| on the left side and binding a label, basically a string, to that node. Then traversing the right side of the operator and looking for \verb|operator+| and also the node which is equal to the already bound node. \medskip
\par Since the checker can be parameterized there is a third method overload named \verb|storeOptions| which handles the passed parameters. If the option \verb|StrictMode| is specified as \verb|true| the checker will be triggered for all inefficient calls, else only for those which are inside a loop. \medskip
\par This way we are sure that the callback will only be called for those operators which have the same variable on the right side as the left side.
\subsubsection{Inefficient Stream Use Check}
\par The implementation fo this checker consists of two distinct parts, one part for matching the character conversions and the other is for matching the multiple uses of \verb|std::endl|. \medskip
\par To match the character conversion, the tool is looking for calls to \verb|operator<<| that have \verb|std::basic_ostream| as one of the arguments, to make sure we are only dealing with cases regarding streams and not interfering with user-defined overloads of \verb|operator<<|. The other argument must be an \verb|ImplicitCastExpr|, that as the name suggests is an implicit cast between two types, which have \verb|const char[2]| as source and \verb|const char *| as a destination. With these structures, we can safely match the inefficient character conversions. \medskip
\par To match multiple \verb|std::endl| use, we can use the left-associative property of \verb|operator<<|. Matching the outermost \verb|std::endl| and looking for any descendant \verb|std::endl| we can make sure only the inner \verb|std::endl| calls will be labelled. \medskip
\par Having constructed the checking functionality so that it only calls the callback function on an exact match we have only formatting to do. In case of \verb|std::endl| the formatting is quite simple, just replace the unwanted \verb|std::endl|s with \verb|'\\n'| to be able to print \verb|'\n\| in the warning message. \medskip
\par In the case of simple characters its a more complex process, because of the character escapes. This is solved by introducing a static free function \verb|getEscapedChar| which contains a character table and maps the characters to their escaped versions.
\subsubsection{Default Container Initialization Check}
\par This module is the most complex in terms of implementation details and structural complexity. To match the smallest scope of code in the \verb|registerMatchers| function we have to introduce several variables. Beforehand as a high-level overview of the AST matcher, we want to define. It should match on such \verb|CompoundStmt|s that contain a \verb|CXXMemberCallExpr|, but are not inside of a template instantiation. \medskip
\par First of all, we define the declarations of the insertion methods \\(\verb|insert, push_back, emplace, emplace_back|), and also the containers we are interested in (\verb|map, vector, set, deque|). Also, we need a matcher that matches \verb|CXXMemberCallExpr|s which reference the container and have the desired insertion declaration. \medskip
\par We need three types of \verb|CXXMemberCallExpr| to match all the possible scenarios. First, we define a variable named \verb|MemberCallExpr| to match \verb|CXXMemberCallExpr| on the previously defined containers, with the specific insertion type. Second we define a variable named \verb|MemberCallExprWithRefToContainer| which is an extension of \verb|MemberExpr| having a reference to the original container as argument, for instance this would match \verb|vec.push_back(vec.size())|. \medskip
\par The third helper variable is \verb|UnresolvedMemberExpr| which matches \\\verb|UnresolvedMemberExpr|. To match this expressions a custom AST matcher had to be created which is specialised on \verb|UnresolvedMemberExpr| and also a matcher that can check the base of such an expression. \medskip
\par The first checker to be added is matching a \verb|CompoundStmt| which is not inside of a template instantiation and matches for each \verb|UnresolvedMemberExpr|, the second matcher is the same as the first matcher but matching a \\\verb|MemberCallExprWithRefToContainer| inside a \verb|CompoundStmt|. The third checker is simply matching \verb|MemberExpr|s inside a \verb|CompoundStmt|. With these three matchers we can pass the callback function optionally matched AST nodes indicating where to stop processing the nodes. \medskip
\par In the callback function we are walking the direct children of the matched \verb|CompoundStmt| til we find the container. We process step by step all the \\\verb|CXXMemberCallExpr|s following the container declaration until there's no more expressions or either the matched \verb|UnresolvedMemberExpr| or \\\verb|MemberCallExprWithRefToContainer| nodes. \medskip
\par For every \verb|CXXMemberCallExpr| we extract the arguments with \\\verb|getInsertionArguments| and format them with \verb|formatArguments|, adding all the required narrowing conversion by calling \verb|getNarrowingCastStr|. After having all the arguments formatted, we create a removal of the insertion expressions with \verb|getRangeWithFollowingToken| which calls the parser to get the next token after the expression. After emitting the diagnostic messages, we add the container to the already processes container declarations.
\subsubsection{Shared Pointer Conversion Check}
\par To summarise in a nutshell the matcher part of the check, it's looking for \verb|CallExpr|s involving such functions that have a \verb|shared_ptr| as an argument and a constructor-conversion is taking place at the place of the function call. \medskip
\par Delving more into the details of the matcher implementation, I should highlight the structure responsible for matching the actual constructor-conversion. To do that we need to define two \verb|QualType|s to represent the type for \verb|shared_ptr| and also \verb|unique_ptr| (I will talk about its importance later) with the help of the function \verb|getQualTypeForTemplate|, which is a C++11 function using tailing return type with \verb|auto| type deduction of \verb|decltype(QualType())|. \medskip
\par Having the aforementioned types, we can construct the part which matches on such \verb|Expr|s which have an \verb|ImplicitCastExpr| that is a \verb|CK_ConstructorConversion| type of cast. Moreover, this cast has to have the type of the previously mentioned \verb|shared_ptr| unless it is a conversion from \verb|unique_ptr| to \verb|shared_ptr|. \medskip
\par With these components, we only need to match such \verb|CallExpr|s that have this implicit conversion and bind a label to it, and also bind a label to the calllee function's declaration. \medskip
\par In the callback function our only task is to resolve the base and the derived type to be able to print a meaningful diagnostic message involving the type names. Getting the derived class's name is relatively easy since the standard specifies\cite[\S20.8.2.2.1]{cpp_standard} that the first template argument of the converting constructor must be the type it's converting from. So we only need to query the first template argument's declared name. \medskip
\par Resolving the name of the base class is not quite this straightforward because the type information is not encoded in such an explicit way in the expressions we already matched. The information about the base type is encoded in the destination type of the implicit cast, which can be retrieved by using the \verb|getBaseTypeAsString| static free function. \medskip
\par Having both types resolved we can print diagnostics as a warning on the actual function call and as a note pointing at the location of the parameter of the callee function, that it should be rewritten to a more efficient code.
\subsection{Program Structure}
\par This section covers the abstract structure of the application in the Clang-tidy project. The visualisation is done with the Unified Modelling Language (UML for short), which is a standard way of representing different logical and physical connections in a computer software. We use UML Class diagrams to visualise the methods and fields of an object, representing also the visibility of these components from the outside of the class.  \medskip
\par In Figure:\ref{inheritence} we visualise each module's ancestor and how these modules are decoupled from each other.
\begin{figure}[H]
 	\caption{Inheritance diagram}
 	\includegraphics[scale=1]{images/inheritence.pdf}
 	\label{inheritence}
\end{figure}
\par Every module extends from the \verb|ClangTidyCheck| class, which provides the interface of a Clang-tidy module to create the AST-matchers and also how the matched code should be handled afterwards. The first step of adding a new module to Clang-tidy is to inherit from this class and implement the methods, which can be used also to send extra parameters to the checkers themselves.
\begin{figure}[H]
	\caption{UML class diagram of Inefficient Stream Use checker}
	\centering
	\includegraphics[scale=1.8]{images/InefficientStreamUse.pdf}
	\label{stream_check_class}
\end{figure}
\par In Figure:\ref{stream_check_class} you can observe one of the simplest module's UML Class diagram, which implements the methods declared by \verb|ClangTidyCheck| and also defines one static free function in the source file.
\begin{figure}[H]
	\caption{UML class diagram of Container Default Initializer check}
	\centering
	\includegraphics[scale=0.85]{images/ContainerDefaultInitializer.pdf}
	\label{container_init_class}
\end{figure}
\par Figure:\ref{container_init_class} represents the abstract structure of the Container Default Initializer check. The solved problem in this module is more complex than the rest hence the increased complexity in the abstract structure. This checker uses several helper functions, and utility classes to store all the meta information about the containers and the conversions taking place in instantiating them.
\begin{figure}[H]
	\caption{UML class diagram of Inefficient String Concatenation check}
	\centering
	\includegraphics[scale=1.8]{images/InefficientStringConcatenation.pdf}
	\label{string_concat_check}
\end{figure}
\par The next module's architecture is fairly simple again, as seen in Figure:\ref{string_concat_check}. It consists of the basic interface of a Clang-tidy checker, but it uses a new method \verb|storeOptions| to be able to receive a passed parameter and modify its behaviour accordingly.
\begin{figure}[H]
	\caption{UML class diagram of Inefficient Shared Pointer check}
	\centering
	\includegraphics[scale=1.8]{images/InefficientSharedPtr.pdf}
	\label{shared_ptr_check}
\end{figure}
\par In Figure:\ref{shared_ptr_check} you can inspect the structure of the Inefficient Shared Pointer checker which declares two extra static methods for retrieving extra information about the matched nodes. 
