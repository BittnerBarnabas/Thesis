\section{Technical background}
\par In this section I will outline pieces of technology related to the main goal of this thesis. Starting from the global framework I'm integrating modules into to the abstract representation of a C++ application.


\subsection{LLVM}
\par LLVM formerly known as Low Level Virtual Machine is a "collection of the reusable and modular compiler technologies"\cite{llvm_mainpage}. LLVM started out as a university project\cite{LLVM:CGO04}, and since then it grew significantly in size and it now offers numerous subprojects which help building and maintining both commercial and open-source applications.
\par The essential goal of LLVM is to provide generalized optimizations to arbitrary programming languages using the LLVM Intermediate Language also known as LLVM IR, which acts as a common representation of different programming languages. This is achieved through using specific language front-ends, which transforms the given language to LLVM IR. 
\begin{figure}[h]
	\caption{LLVM workflow}
	\includegraphics[scale=0.33]{images/llvm_flow}
\end{figure}

\subsection{Clang}
\par The most popular C/C++ familiy compiler front-end for LLVM is Clang which aims to excel from the open-source compilers with it exceptionally fast compile-times and user-friendly diagnostic messages. 
\subsection{Clang-tidy}
\subsection{Abstract syntax tree - AST}